\documentclass[11pt]{aghdpl}
% \documentclass[en,11pt]{aghdpl}  % praca w języku angielskim
\usepackage[polish]{babel}
%\usepackage[english]{babel}
\usepackage[utf8]{inputenc}

% dodatkowe pakiety
\usepackage{enumerate}
\usepackage{listings}
\lstloadlanguages{TeX}

\lstset{
  literate={ą}{{\k{a}}}1
           {ć}{{\'c}}1
           {ę}{{\k{e}}}1
           {ó}{{\'o}}1
           {ń}{{\'n}}1
           {ł}{{\l{}}}1
           {ś}{{\'s}}1
           {ź}{{\'z}}1
           {ż}{{\.z}}1
           {Ą}{{\k{A}}}1
           {Ć}{{\'C}}1
           {Ę}{{\k{E}}}1
           {Ó}{{\'O}}1
           {Ń}{{\'N}}1
           {Ł}{{\L{}}}1
           {Ś}{{\'S}}1
           {Ź}{{\'Z}}1
           {Ż}{{\.Z}}1
}

%---------------------------------------------------------------------------

\author{Mateusz Kulpa}
\shortauthor{M. Kulpa}

\titlePL{TODO Projekt i~implementacja aplikacji do wspomagania pracy gabinetu lekarskiego w technologii JavaEE oraz Android}
\titleEN{TODO ENGLISH TITLE}

\shorttitlePL{Skrócony tytuł polski} % skrócona wersja tytułu jeśli jest bardzo długi
\shorttitleEN{Skrócony tytuł angielski}

\thesistype{Praca dyplomowa magisterska}

\supervisor{dr inż.\ Paweł Skrzyński}

\degreeprogramme{Informatyka}


\date{2014}

\department{Katedra Informatyki Stosowanej}
%\department{Department of Applied Computer Science}

\faculty{Wydział Elektrotechniki, Automatyki,\protect\\[-1mm] Informatyki i Inżynierii Biomedycznej}
%\faculty{Faculty of Electrical Engineering, Automatics, Computer Science and Biomedical Engineering}

\acknowledgements{Serdecznie dziękuję \dots tu ciąg dalszych podziękowań np. dla promotora, żony, sąsiada itp.}


\setlength{\cftsecnumwidth}{10mm}

%---------------------------------------------------------------------------
\setcounter{secnumdepth}{4}

\begin{document}

\titlepages
\setcounter{tocdepth}{3}
\tableofcontents
\clearpage

\chapter{Abstrakt}

%----------------------------------------------------------------------------

\chapter{Wstęp}
\label{cha:wstep}

\section{Cele pracy}
\label{sec:celePracy}

\section{Zawartość pracy}
\label{sec:zawartoscPracy}

%----------------------------------------------------------------------------

\chapter{Architektura}

\section{Architektura fizyczna}
\section{Architektura logiczna}

%----------------------------------------------------------------------------

\chapter{Dobór technologii i narzędzi}

\section{Baza Danych}
\subsection{System zarządzania relacyjnymi bazami danych}
\subsection{Mapowanie obiektowo-relacyjne}
\section{Web Service}
\subsection{Protokół}
\subsubsection{SOAP}
\subsubsection{REST}
\subsubsection{Wybór i uzasadnienie}
\subsection{Serializacja}
jersey/jackson
\section{Moduł web}
\subsection{backend}
playframework
\subsection{frontend}
bootstrap, javascript, jquery, html, css
\section{Moduł mobile}
\subsection{Platforma}
Android
\subsection{Klient WebService}
ApacheHttpClient

%----------------------------------------------------------------------------

\chapter{Implementacja}

\section{Baza danych}
\subsection{Encje}
\subsection{Bezpieczeństwo}
\subsubsection{SQL Injection}
\section{Web Service}
\subsection{API}
"rozdwojenie" API dla web i mobile
zalecany interfejs restowy a ograniczenia html: (brak PUT, DELETE), paginacja w url czy header Range?
\subsection{Bezpieczeństwo}
\subsubsection{HTTPS}
\subsubsection{Ochrona wybranych zasobów}
zwykłe podejście czyli chroniony url vs adnotacje
@Secured, przekierowania, sesja (krótko, więcej w osobnym rozdziale)
\section{Autentykacja}
\subsection{Różne podejścia}
basic auth, form auth itp...
\subsection{Bezpieczeństwo}
\subsubsection{Hashowanie}
różne algorytmy i ich podatność na ataki
hashowanie u klienta czy po stronie serwera?
długość klucza
salt
rodzaje, bezpieczeństwo, haslo przesylane clear text vs md5 robione u klienta
\subsubsection{HTTPS}
clear text
\section{Reset hasła}
tokeny, mail
\section{Sesja}
session key (dlugosc, bezpieczenstwo, SecureRandom CSPRNG) w bazie czy encryption? encryption = zysk na perf i stateless.. baza może być minusem ale tu i tak jest konieczna do przechowywania innych danych

%----------------------------------------------------------------------------

\chapter{Opis}

\section{Web}
\subsection{Zakres}
\subsection{Prezentacja}
\section{Mobile}
\subsection{Zakres}
\subsection{Prezentacja}
\section{User Experience}
układ formularzy, komunikaty po akcjach (do zrobienia?), walidacja pól, kalendarz, wyszukiwarka, paginacja, (walidacja PESEL??)
%----------------------------------------------------------------------------

\chapter{Konfiguracja i uruchomienie}
a może jako dodatek zamiast rozdziału?

\subsection{Baza danych}
\subsection{Play Framework}
\subsubsection{Instalacja}
\subsubsection{Plik application.conf}
baseUrl, smtp.host, database itp...
\subsection{Android SDK}
\subsection{Zależności}
\subsubsection{Sbt}
\subsubsection{Gradle}
\subsection{Uruchomienie}

% itd.
% \appendix
% \include{dodatekA}
% \include{dodatekB}
% itd.

\chapter{Bibliografia} %czy chapter jest potrzebny?

\bibliographystyle{alpha}
\bibliography{bibliografia}
%\begin{thebibliography}{1}
%
%\bibitem{Dil00}
%A.~Diller.
%\newblock {\em LaTeX wiersz po wierszu}.
%\newblock Wydawnictwo Helion, Gliwice, 2000.
%
%\bibitem{Lam92}
%L.~Lamport.
%\newblock {\em LaTeX system przygotowywania dokumentów}.
%\newblock Wydawnictwo Ariel, Krakow, 1992.
%
%\bibitem{Alvis2011}
%M.~Szpyrka.
%\newblock {\em {On Line Alvis Manual}}.
%\newblock AGH University of Science and Technology, 2011.cccccc
%\newblock \\\texttt{http://fm.ia.agh.edu.pl/alvis:manual}.
%
%\end{thebibliography}

\end{document}
