\documentclass[11pt]{aghdpl}
% \documentclass[en,11pt]{aghdpl}  % praca w języku angielskim
\usepackage[polish]{babel}
%\usepackage[english]{babel}
\usepackage[utf8]{inputenc}

% dodatkowe pakiety
\usepackage{enumerate}
\usepackage{listings}
\lstloadlanguages{TeX}

\lstset{
  literate={ą}{{\k{a}}}1
           {ć}{{\'c}}1
           {ę}{{\k{e}}}1
           {ó}{{\'o}}1
           {ń}{{\'n}}1
           {ł}{{\l{}}}1
           {ś}{{\'s}}1
           {ź}{{\'z}}1
           {ż}{{\.z}}1
           {Ą}{{\k{A}}}1
           {Ć}{{\'C}}1
           {Ę}{{\k{E}}}1
           {Ó}{{\'O}}1
           {Ń}{{\'N}}1
           {Ł}{{\L{}}}1
           {Ś}{{\'S}}1
           {Ź}{{\'Z}}1
           {Ż}{{\.Z}}1
}

%---------------------------------------------------------------------------

\author{Mateusz Kulpa}
\shortauthor{M. Kulpa}

\titlePL{Projekt i~implementacja mobilnego systemu wspomagającego prowadzenie gabinetu lekarskiego w technologii JavaEE oraz Android}
\titleEN{Design and implementation of mobile doctors office management system using JavaEE and Android technologies}

\shorttitlePL{Projekt i~implementacja mobilnego systemu wspomagającego prowadzenie gabinetu lekarskiego w technologii JavaEE oraz Android}
\shorttitleEN{Skrócony tytuł angielski}

\thesistype{Praca dyplomowa magisterska}

\supervisor{dr inż.\ Paweł Skrzyński}

\degreeprogramme{Informatyka}

\date{2014}

\department{Katedra Informatyki Stosowanej}
%\department{Department of Applied Computer Science}

\faculty{Wydział Elektrotechniki, Automatyki,\protect\\[-1mm] Informatyki i Inżynierii Biomedycznej}
%\faculty{Faculty of Electrical Engineering, Automatics, Computer Science and Biomedical Engineering}

\acknowledgements{Serdecznie dziękuję \dots tu ciąg dalszych podziękowań np. dla promotora, żony, sąsiada itp.}


\setlength{\cftsecnumwidth}{10mm}

%---------------------------------------------------------------------------

%TODO Content Negotiation
%TODO StringEntity entity = new StringEntity(mapper.writeValueAsString(patient), HTTP.UTF_8);
%TODO Maintenece - spójność api - latwiejsze w utrzymaniu, klient ma tylko widok i pobiera dane z ws łącznie ze zdefiniowanymi leczeniami/rozpoznaniami. Dzięki temu nie trzeba będzie aktualizować obydwu aplikacji, a jedynie zdefiniować nowe leczenie/rozpoznanie po stronie serwera. Gdyby api było zewnętrzne to warto by się pokusić o oddzielny interjefs, skoro jest tylko wew. wygodniej będzie utrzymywać jeden
%TODO ajax, jquery?
%TODO play - formularze z zagnieżdżonymi obiektami oraz listami

\setcounter{secnumdepth}{4}

\begin{document}


\titlepages

\begin{abstract}

Niniejsza praca opisuje etapy implementacji narzędzia wspomagającego pracę gabinetu lekarskiego. Głównym podmiotem pracy nie jest stworzenie kompletnego rozwiązania dla lekarzy zgodnego z~ obowiązującymi normami, a przedstawienie sposobu implementacji systemu składającego się  z~kilku komponentów, z~którego można korzystać przy pomocy przeglądarki internetowej oraz aplikacji dedykowanej na platformę Android. Ze względu na duże zróżnicowanie potrzeb w zależności od dziedziny medycyny zdecydowano się na implementację narzędzia przeznaczonego dla gabinetu dentystycznego. 

W pracy poruszane są kwestie projektowania aplikacji biznesowej pod kątem dostępnych obecnie technologii z~naciskiem na interfejsy poszczególnych warstw. Szczegółowo opisywane są również typowe problemy bezpieczeństwa z~jakimi trzeba się zmagać podczas implementacji poszczególnych modułów takie jak uwierzytelnianie, bezpieczny mechanizm resetu hasła czy SQL Injection.

Pomimo zawężenia problemu do wielkości pozwalającej na skupienie się na szczegółach projektowania i~ implementacji, opisywany system realizuje w podstawowym stopniu potrzeby związane z prowadzeniem przychodni stomatologicznej i~może zostać dlań wykorzystany.

\end{abstract}

\setcounter{tocdepth}{3}
\tableofcontents
\clearpage

%----------------------------------------------------------------------------

\chapter{Wstęp}
\label{cha:wstep}
%TODO problem spójności api
%TODO praca odnosi się do konkretnych frameworków i niektóre rozwiązania są do zastosowania tylko w ich obrębie, ale niektóre zagadnienia mają charakter ogólny
\section{Cele pracy}
\label{sec:cele_pracy}

\section{Zawartość pracy}
\label{sec:zawartosc_pracy}

%----------------------------------------------------------------------------

\chapter{Architektura}
\label{cha:architektura}

\section{Architektura fizyczna}
\label{sec:architektura_fizyczna}

\section{Architektura logiczna}
\label{sec:architektura_logiczna}

%----------------------------------------------------------------------------

\chapter{Dobór technologii i narzędzi}
\label{cha:dobor_technologii_i_narzedzi}

W rozdziale tym opisane są technologie~i narzędzia użyte do implementacji każdej~z warstw aplikacji. Nie uwzględniono przy tym bibliotek zastosowanych do rozwiązania konkretnych problemów dla danej warstwy, a jedynie elementy niezbędne do zrozumienia struktury systemu~i sposobu komunikacji pomiędzy poszczególnymi modułami.

%TODO dopisać coś o Javie

\section{Baza Danych}
\subsection{System zarządzania relacyjnymi bazami danych}
\subsection{Mapowanie obiektowo-relacyjne}
\section{Web Service}
\subsection{Protokół}
\subsubsection{SOAP}
\subsubsection{REST}
\subsubsection{Wybór i uzasadnienie}
\subsection{Serializacja}
jackson
\section{Moduł web}
\subsection{Back-end}
playframework
\subsection{Front-end}
bootstrap, javascript, jquery, html, css
\section{Moduł mobile}
\subsection{Platforma}
Android
\subsection{Klient WebService}
ApacheHttpClient
\section{Zarządzanie zależnościami}
\subsection{Sbt}
\subsection{Gradle}

%----------------------------------------------------------------------------

\chapter{Implementacja}
\label{cha:implementacja}

\section{MVC}

\subsection{Modele}

\subsection{Widoki}

%jedyne miejsce zmienne dla urządzeń!!

\subsubsection{Web}

\subsubsection{Mobile}

\subsection{Kontrolery}

%TODO CRUD, transakcje, acid?

\section{Web Service}

\subsection{API}

``rozdwojenie'' API dla web i mobile

zalecany interfejs restowy a ograniczenia html: brak PUT, DELETE; paginacja w url czy header Range?

\section{Bezpieczeństwo}

\subsection{Uwierzytelnianie}

%co prawda jest po stronie web ale dotyczy jednak obydwu modułów

\subsubsection{Różne podejścia}

basic auth, form auth itp...

wysyłanie clear text, hashowanie na serwerze

\subsubsection{Hashowanie}

różne algorytmy i ich podatność na ataki

hashowanie u klienta czy po stronie serwera?

długość klucza

salt

rodzaje, bezpieczeństwo, haslo przesylane clear text vs md5 robione u klienta

\subsubsection{Sesja}

session key (dlugosc, bezpieczenstwo, SecureRandom CSPRNG) w bazie czy encryption? encryption = zysk na perf i stateless.. baza może być minusem ale tu i tak jest konieczna do przechowywania innych danych

\subsection{Reset hasła}

tokeny, mail

\subsection{Ochrona wybranych zasobów}

zwykłe podejście czyli chroniony url (trochę o podejściu klasycznym tzn. ochrona url bez wiedzy aplikacji z logowaniem jako osobny niechroniony moduł) vs adnotacje
@Secured, przekierowania, sesja (krótko, więcej w osobnym rozdziale)

\subsection{HTTPS}

\subsection{SQL Injection}

\subsection{XSS}


%----------------------------------------------------------------------------

\chapter{Opis}
\label{cha:opis}

\section{Web}
\subsection{Zakres}
\subsection{Prezentacja}
\section{Mobile}
\subsection{Zakres}
\subsection{Prezentacja}
\section{User Experience}
układ formularzy, komunikaty po akcjach (do zrobienia?), walidacja pól, kalendarz, wyszukiwarka, paginacja, (walidacja PESEL??)
%----------------------------------------------------------------------------

\chapter{Konfiguracja i uruchomienie}
\label{cha:konfiguracja_i_uruchomienie}
a może jako dodatek zamiast rozdziału?

\subsection{Baza danych}
\subsection{Play Framework}
%pamiętaj o https
\subsubsection{Instalacja}
\subsubsection{Plik application.conf}
baseUrl, smtp.host, database itp...
\subsection{Android SDK}
\subsection{Zależności}
\subsubsection{Sbt}
\subsubsection{Gradle}
\subsection{Uruchomienie}

% itd.
% \appendix
% \include{dodatekA}
% \include{dodatekB}
% itd.

\chapter{Bibliografia} %czy chapter jest potrzebny?
\label{cha:bibliografia}

\bibliographystyle{alpha}
\bibliography{bibliografia}
%\begin{thebibliography}{1}
%
%\bibitem{Dil00}
%A.~Diller.
%\newblock {\em LaTeX wiersz po wierszu}.
%\newblock Wydawnictwo Helion, Gliwice, 2000.
%
%\bibitem{Lam92}
%L.~Lamport.
%\newblock {\em LaTeX system przygotowywania dokumentów}.
%\newblock Wydawnictwo Ariel, Krakow, 1992.
%
%\bibitem{Alvis2011}
%M.~Szpyrka.
%\newblock {\em {On Line Alvis Manual}}.
%\newblock AGH University of Science and Technology, 2011.cccccc
%\newblock \\\texttt{http://fm.ia.agh.edu.pl/alvis:manual}.
%
%\end{thebibliography}

\end{document}
